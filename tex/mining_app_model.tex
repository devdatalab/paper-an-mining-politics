\doublespacing

In this section, we present a political agency model to elucidate the
channels by which resource extraction operations influence the
behavior of politicians. The model is in the spirit of the career
concerns model of \citeasnoun{Persson2000}, which
\citeasnoun{Brollo2009} extend to allow endogenous entry of
politicians. Our model is oriented toward understanding rent-seeking
through illegal collusion between politicians and firms.

We focus on collusion between politicians and firms because that
relationship is suggested by the other literature on mining firms in
India and indeed in many other places around the world. Models of
politicians and firms have more commonly focused on the case where
politicians extract rents from firms to the detriment of those firms
\cite{Shleifer1994}. A model in that spirit would generate similar
predictions to those that we describe below, but it fits less well
with the qualitative evidence from India. Further, the facilitation of
black markets is widely associated in the literature with the presence
of violent actors
\cite{Tilly1985,Gambetta1996,Bandiera2003,Chimeli2011,Skarbek2011}.

We focus on two features of the resource extraction sector. First, the
mining sector generates rents, which can be expropriated by
politicians through their control over the regulatory inputs required
by mining firms. Second, mining is rife with illegality, both in India
and in other developing countries. This increases the dependence of
firms on local authorities, and raises the relative returns to both
politicians and firms willing to engage in illegal activity.
  
Consider a single mining firm that operates in a constituency
represented by a single politician. The mining operation has a high
fixed cost and a low marginal cost; the price of output is such that
the firm is profitable.  Politicians have a type that is characterized
by returns to illegal behavior $\theta \in (0,1)$. A high $\theta$
could represent a low risk aversion, indicating a willingness to risk
being caught and punished for crime. It could also represent a set of
skills that increase returns from criminal activities, such as a
propensity toward violence, or connections to criminal networks and
other corrupt officials.\footnote{We do not take a stand in the model
  on the relationship between $\theta$ and the politician's ability to
  provide services to constituents. \citeasnoun{Brollo2009} assume
  that corrupt politicians provide worse services to citizens;
  \citeasnoun{Vaishnav2017} argues they are better at providing
  services, in part because the formal state does such a poor job.}
Politicians who are caught in illegal activities pay a formal legal
punishment and may face worse odds of re-election.\footnote{While we
  view it as unlikely that voters would reward a politician for being
  convicted, the model only requires that the punishment from being
  caught outweighs any electoral benefit.}

We intentionally treat $\theta$ as a generally propensity toward crime
that is not specific to any type of crime, as this appears to fit the
context. Qualitative evidence suggests that a willingness to commit
crimes for one's party organization or local bosses is used as an
intentional signaling strategy. Such politicians may wish it to be
known that they are effective at acting outside of the law
\cite{Witsoe2009a,Berenschot2011a,Vaishnav2017}.

The model has two periods. In the first period, each candidate
chooses an election campaign effort level $e$, with a convex cost
$f(e)$. This could be a time cost or a financial cost. Election
outcomes cannot be predicted with certainty and the probability of
getting elected is a concave function of effort, which we denote
$\pi(e)$. The candidate's utility function is:

\begin{equation}
U = \pi(e)g(\cdot) - f(e)\text{,}
\end{equation}

\noindent where $g(\cdot)$ is the utility gain from getting
elected, and includes the continuation value of future elections.

In the second period, in exchange for payment, the elected politician
can take an illegal action that increases the firm's output, such as
granting an environmental clearance or land use permit that would have
been rejected by the formal process.\footnote{While the action itself
  may be legal or illegal, the exchange of the action for payment is
  illegal. Other actions could be expediting a permit that would have
  been granted anyway (a less serious crime), or arranging for police
  to arrest or intimidate local activists (a more serious crime).}
The action raises the firm's output by an increasing concave function
$q(a)$; more serious crimes (with higher $a$) have bigger effects on
output.\footnote{Any crimes for which the marginal profit is not
  increasing in the severity of the crime would be dominated choices,
  and thus not considered. We could model criminal competency by
  assuming that $q()$ is a positive function of $\theta$; this
  strengthens the predictions below because the politician trades off
  the increase in $q()$ against the cost of crime, which is decreasing
  in $\theta$.}  The action increases the firm's profit by $\mu q(a)$,
where $\mu$ is the mineral markup, or the difference between the price
and extraction cost of the mineral. If the politician takes the
illegal action, the rents are shared according to the Nash Bargaining
solution. We assume equal discount rates for simplicity but the model
results do not depend on this assumption, as long as the difference in
discount rates is not extreme.  The utility cost of illegal action is
$\frac{c(a)}{\theta}$, where $c()$ is a convex increasing function of
the severity of the action $a$.  The cost function encapsulates the
probability of being caught, the punishment conditional upon being
caught, and any future electoral consequences. High $\theta$
politicians pay a lower utility cost for committing a given crime.

Equation~\ref{eq:gain} summarizes the politician's net utility from
the illegal action:
%
\begin{equation}
\label{eq:gain}
g(a,\mu,\theta) = \frac{1}{2} \big(\mu q(a) - \frac{c(a)}{\theta}\big)\text{.}
\end{equation}

\noindent We solve the model by backward induction. In the second
period, the politician
chooses $a$ to maximize rents, trading off profit against the risk and
cost of
getting caught. The first order condition is:
%
\begin{equation}
\label{eq:foc1}
\mu q'(a^*)=\frac{c'(a^*)}{\theta}\text{.}
\end{equation}

\noindent Under Inada conditions, any politician with $\theta$ strictly greater
than zero will choose $a^* > 0$ and commit at least some illegal
action.\footnote{In the words of a four-time Chief Minister of Uttar
  Pradesh, ``Even an honest MLA [politician] gets a [10\%] kickback on
  discretionary spending'' \cite{Vaishnav2017}.}

If the price of mineral output, and thus the mineral markup $\mu$
rises, then crime severity $a^*$ must rise according to
Equation~\ref{eq:foc1}.\footnote{Specifically,
  $\frac{\partial{a^*}}{\partial{\mu}} =
  \frac{q'(a^*)}{\frac{c''(a^*)}{\theta}-\mu q''(a^*)}$.}
Since $\mu$ and $a^*$ are rising, the
politician's rents in Equation~\ref{eq:gain} must rise as
well.  This gives us the moral hazard result: when mineral rents are
high, politicians provide more illegal services to firms and both
firms and politicians earn greater rents from mining operations.

We now consider how politician type affects the effort exerted to
obtain office. Each candidate chooses an effort level such that the
marginal gain in terms of rents in office is equal to the marginal
cost of effort required to win:
%
\begin{equation}
f'(e^*) = \frac{1}{2} \pi'(e^*) \big( \mu q(a^*) - \frac{c(a^*)}{\theta}\big)\text{.}
\end{equation}
%
The change in effort in response to an increase in the mineral markup
is given by:
%
\begin{equation}
\frac{\partial e^*}{\partial \mu} = \frac{1}{2} \frac{\pi'(e^*)\cdot
  q(\mu,\theta)}{f''(e^*)-\pi''(e^*)\cdot g(\mu,\theta)}\text{.}
\end{equation}

\noindent The expression is positive. Politicians earn greater rents from office
when mineral rents are high, and therefore all candidates try
harder to win elections when mineral prices are high.
This has the biggest effect on effort when $q(\cdot)$ is large, and
thus when $\theta$ is large---that is, on the candidates with the
highest propensity toward illegal activity.  High mineral values will
increase the probability that a criminally inclined candidate gets
elected, unless the mineral shock also decreases voters' preferences
for criminal candidates. This is the adverse selection
effect.\footnote{For simplicity, we have assumed that mineral wealth
  does not affect voter preferences over candidate type. Voter
  preferences could shift in either direction. They may dislike
  criminal candidates, and pay closer attention to elections when
  rents are high, thus mitigating the adverse selection
  effect. Alternately, they may prefer criminal candidates if they are
  perceived to facilitate mining operations.  In the empirical part of
  this paper, we observe a joint outcome of voter preferences and
  candidate effort. The empirical test of the selection effect is thus
  jointly testing for the sum of the increase in candidate effort and
  any voter shift \textit{toward} the more criminal candidate.}  Both
the adverse selection and moral hazard effects lead to increased
illegal behavior by politicians in office when mineral rents are
high. These effects are likely not only additive, but reinforcing: the
moral hazard effect is worse for candidates who are more criminally
inclined. 

The model makes three key predictions, which we test in this
paper. First, positive mineral wealth shocks in the first period (i.e.
before elections take place) will lead criminal politicians to win
more elections. Second, positive mineral wealth shocks in the second
period (i.e. after candidates have been selected into office) will
cause politicians in office to gain wealth and commit more crimes. By
focusing on shocks that occur after candidates win elections, we can
thus isolate the moral hazard effect. Third, the wealth and crime
gains may occur for all types of politicians, but should be strongest
for the most criminal types.


\subsection{Modeling Electoral Fraud}
\label{app:model_fraud}

This subsection extends the model by considering the possibility that
politicians can use criminal activities or violence to win elections.

The structure is as before, but we assume that crime with the purpose
of winning elections, denoted by $a_e$, is distinctly chosen from
crime to increase mining output, which we call illegal mining and
denote by $a_m$. Both are punished with the same convex increasing
cost function $\frac{c(a)}{\theta}$. As above, we use backward
induction. We first solve for second period rents and criminal behavior
conditional on winning an election. Then, we solve for electoral
effort and electoral crime in the first period.

The elected politician earns the following rent, which is unchanged
from the model above:

\begin{equation}
g(a_m,\mu,\theta) = \frac{1}{2} \big(\mu q(a_m) -
\frac{c(a_m)}{\theta}\big)\text{.}
\end{equation}

The first order condition for the extent of illegal mining is unchanged:
\begin{equation}
\mu q'(a_m^*)=\frac{c'(a_m^*)}{\theta}\text{.}
\end{equation}

The politician's utility function is as follows. We add a
choice over electoral crime, and a cost function for electoral crime.

\begin{equation}
U = \pi(e,a_e)\frac{1}{2} \big(\mu q(a_m) -
\frac{c(a_m)}{\theta}\big) - f(e) - \frac{c(a_e)}{\theta}\text{.}
\end{equation}

As in the primary model, $e$ represents effort to win elections, $a_m$
is the extent of illegal mining, $\mu$ is the mineral markup, $q()$ is
the output from illegal mining activities,
$\theta$ is a measurement of propensity toward crime, and $f(e)$ is
the convex cost of electoral effort. We have added additional terms
$a_e$, which denotes the extent of electoral fraud, and $c(a_e)$, the
convex cost of electoral fraud, which incorporates both the
probability of getting caught and the utility punishment. The
probability of winning an election $\pi{e,a_e}$ now depends positively
on effort and electoral crime. This function is concave in both $e$
and $a_e$, and we assume for simplicity that the cross-partial
$\pi_{ea_e}$ is zero.\footnote{If the cross-partial is not zero, it is
  most likely positive, as investment in the capacity to commit one
  kind of crime (e.g. by hiring thugs or bribing police officers)
  likely lowers the cost of committing other crimes. A positive
  cross-partial derivative would further increase the adverse
  selection effect because it raises the return to electoral
  crime for politicians already involved in illegal mining.}

Candidates now jointly choose electoral effort and electoral
crime. The first order conditions are similar for these two choices,
but the amount of electoral crime depends directly on politician type:
\begin{equation}
\frac{\partial f}{\partial e^*} = \frac{1}{2} \frac{\partial \pi}{\partial e^*}
\big( \mu q(a_m^*) - \frac{c(a_m^*)}{\theta}\big)
\end{equation}


\begin{equation}
\frac{1}{\theta} \frac{\partial c}{\partial a_e^*} = \frac{1}{2} \frac{\partial
  \pi}{\partial a_e^*} \big( \mu q(a_m^*) -
\frac{c(a_m^*)}{\theta}\big)
\end{equation}

The moral hazard effect remains unchanged, because the decision about
how much illegal mining to facilitate happens only conditional upon
having been elected:

\begin{equation}
\frac{\partial a_m^*}{\partial \mu} =
\frac{q'(a_m^*)}{\frac{c''(a_m^*)}{\theta}-\mu q''(a_m^*)}
\end{equation}
\newline

There are now two adverse selection comparative statics. When the
mineral markup $\mu$ rises, candidates can change their effort levels,
and they can change their willingness to engage in electoral
crime. The expressions
for these comparative statics are calculated from the election effort
and crime first order conditions:

\begin{equation}
\frac{\partial e^*}{\partial \mu} = \frac{1}{2} \frac{\frac{\partial
    \pi}{\partial e^*}q(a_m^*)}{f''(e^*)-\frac{\partial^2\pi}{\partial e^*^2}
  g(a_m^*,\mu,\theta)}
\end{equation}

\begin{equation}
\frac{\partial a_e^*}{\partial \mu} = \frac{1}{2} \frac{\frac{\partial
    \pi}{\partial a_e^*} q(a_m^*)}{\frac{1}{\theta} \frac{\partial^2
    c}{\partial a_e^*^2} - \frac{ \partial^2 \pi}{\partial a_e^*^2}
  g(a_m^*,\mu,\theta)}
\end{equation}

The first expression is unchanged. The second expression demonstrates
a second form of adverse selection: mineral rents increase the return
to electoral crime, and do so especially for high $\theta$
politicians. This occurs because these politicians facilitate
more illegal mining $q(a_m^*)$ and thus have greater marginal returns
to crime when prices are high.

In conclusion, extending the model to give politicians the opportunity
to commit crimes to win elections strengthens the predictions on the
adverse selection effect.

%Summary Stats
 \begin{table}[H]\caption{Summary statistics}
   \small \begin{tabular}{l r r r r r}\hline\hline

           &      &      &   & \multicolumn{2}{c}{\underline{Balance Test}} \\ 
  Variable & Mean & S.D. & N & Beta_{ps} & SE_{ps}  \\ \hline

\multicolumn{6}{l}{\textbf{Adverse Selection Cross-Section: 399 Constituencies}} \\
\quad Number deposits                          & 3.01          & 3.73          & 399          &  & \\
\quad Average annual mineral output (1000 USD) & 8470          & 29718          & 399          &  & \\
\quad People per Primary School                & 1183       & 655       & 399       &  & \\
\quad Constituency Population                  & 247630              & 141909              & 399              &  & \\
\quad Rural Population Share                   & 0.81      & 0.22      & 399      &  & \\
\quad Share Villages with Electricity          & 0.88 & 0.22 & 399 &  & \\

\hline
\multicolumn{6}{l}{\textbf{Adverse Selection Time Series: 948 Constituency-Years)}} \\
\quad Representative Faces Charges                                      & 0.33 & 0.47 & 948 & 0.060 & 0.06 \\
\quad Share Candidates Facing Charges                                   & 0.19   & 0.20   & 948   & 0.030   & 0.03   \\
\quad Representative Faces Violent Charges                              & 0.09       & 0.29       & 938       & -0.030       & 0.04       \\
\quad Representative Faces Corruption Charges                           & 0.08       & 0.26       & 938       & -0.010       & 0.05       \\
\quad INC Representative                                                & 0.32           & 0.47           & 863           & -0.030           & 0.06           \\
\quad BJP Representative                                                & 0.34           & 0.47           & 863           & 0.060           & 0.04           \\
\quad Representative High School Graduate                               & 0.75            & 0.43            & 918            & -0.060            & 0.06            \\
\quad Representative Age                                                & 49.17           & 9.73           & 948           & 0.740           & 1.12           \\
\quad Representative Log Net Assets                                     & 15.75        & 2.82        & 948        & 0.000        & 0.50        \\
\quad Effective Number of Parties                                       & 2.97             & 0.71             & 528             & -0.060             & 0.08             \\
\quad Election Turnout                                                  & 0.69              & 0.08              & 447              & 0.020              & 0.01              \\
\quad Incumbent Winner                                                  & 0.44                  & 0.50                  & 664                  & 0.16**                  & 0.07                  \\
\quad Win Margin                                                        & 0.11              & 0.09              & 863              & 0.000              & 0.01              \\
 \qquad \textit{p-value from F test of joint significance:  0.66} &                               &                                 &                            &                            &                             \\
 \hline
 
\multicolumn{6}{l}{\textbf{Moral Hazard Time Series: 629 Candidates}} \\
\quad Log Net Assets (USD) (first term)     & 15.73 & 2 & 696 & & \\
\quad Log Net Assets (USD) (second term)    & 16.70 & 1 & 696 & & \\
\quad Log Asset Change (balance test)        &                &                    &             & -0.050 & 0.28    \\
\quad Facing Criminal Charges (first term)  & 1.28  & 4  & 629  & & \\
\quad Facing Criminal Charges (second term) & 0.97  & 2  & 629  & & \\
\quad Log Crime Change (balance test)        &                &                    &             & 0.060 & 0.13    \\
\hline\end{tabular}

   \label{tab:sumstat}
   \newline The table presents mean values for all variables used. The
   final two columns show coefficient and standard errors from a
   regression of the row variable on a forward-looking price shock
   (\textit{i.e.}, the shock that occurs \textit{after} the value is
   measured). The estimating equation is
   $Y_{c,d,s,t} = \beta_0 + \beta_1 *
   PriceShock_{c,d,s,t+1,t+6}+\gamma_{s,t} + \nu_{d}+\epsilon_{c,d,s,t}$,
   where $t$ is the first period where a given outcome can be observed.
   All regressions include state-year fixed effects and constituency
controls for the number of deposits within 10km of a constituency, a
constituency-level mineral dispersion index, and baseline (2001)
values of log constituency population, share of the population living
in rural areas, share of villages with electricity and the per
capita number of primary schools.  Standard errors are robust and
clustered at the district level.%

 \end{table}

%%%%%%%%%%%%%%%%%%%%
%%PRICE SHOCK TABLES
%%%%%%%%%%%%%%%%%%%%

%%%% ADR 

%% Winner Any Crim
\newpage
\begin{table}[H]\caption{Effect of mineral price shocks on winning candidate criminality}
  \newcommand{\tablenote}{The table estimates the impact of a local
    mineral price shock on the criminality of the local elected
    politician. The price shock is the change in global mineral prices, weighted by
constituency pre-sample production values of each mineral, calculated
over the five years preceding the given election.%
 The dependent variable is an
    indicator that takes the value one if the local election winner is
    facing criminal charges. Column 1 estimates Equation~\ref{eq:main}
    on the full sample with state*year fixed effects. Columns 2 and 3
    respectively add district and constituency fixed effects. Sample
    size falls because constituency boundaries were redefined in 2007
    and there is only one observation per constituency for most
    predelimitation boundaries. Column 4 shows the marginal effect from
    a probit estimation of a similar specification to that in Column
    1. All regressions include state-year fixed effects and constituency
controls for the number of deposits within 10km of a constituency, a
constituency-level mineral dispersion index, and baseline (2001)
values of log constituency population, share of the population living
in rural areas, share of villages with electricity and the per
capita number of primary schools.  Standard errors are robust and
clustered at the district level.%
 }
  \small \setlength{\linewidth}{.1cm} \begin{center}
\newcommand{\contents}{\begin{tabular}{l*{4}{c}}
\hline\hline
                    &         (1)   &         (2)   &         (3)   &         (4)   \\
\hline
Price shock$_{-6,-1}$&       0.114***&       0.097** &       0.118*  &       0.114***\\
                    &     (0.044)   &     (0.044)   &     (0.069)   &     (0.042)   \\
\hline State-Year F.E.      & Yes & Yes & Yes & Yes \\ 
 District F.E.        & No  & Yes & Yes & No  \\ 
 Constituency F.E.    & No  & No  & Yes & No  \\ 
\hline 
Mean Dep. Var. & 0.33 & 0.33 & 0.32 & 0.33 \\ 
%PYTHON_FOOTER
\hline
N                   &         948   &         946   &         729   &         896   \\
r2                  &        0.15   &        0.35   &        0.67   &               \\
\hline
\multicolumn{5}{p{\linewidth}}{$^{*}p<0.10, ^{**}p<0.05, ^{***}p<0.01$} \\
\multicolumn{5}{p{\linewidth}}{\footnotesize \tablenote}
\end{tabular} }
\setbox0=\hbox{\contents}
\setlength{\linewidth}{\wd0-2\tabcolsep-.25em} \contents \end{center}

  \label{tab:winner_crim}
\end{table}

\newpage
\begin{table}[H]\caption{Effect of mineral price shocks on other winning candidate
    characteristics}
  \label{tab:ed_age}
  \newcommand{\tablenote}{ This table estimates the impact of a
    local mineral price shock on characteristics of the local elected
    leader (as in Table \ref{tab:winner_crim}). The price shock is the change in global mineral prices, weighted by
constituency pre-sample production values of each mineral, calculated
over the five years preceding the given election.%
 The dependent
    variable in the five columns is as follows: (1) an indicator
    that takes the value of one if the winner is a member of the
    Bharatiya Janata Party (BJP); (2) an indicator that takes the value one if he/she is a
    member of the Indian National Congress (INC) party; (3) an
    indicator that takes the value one if the winner has completed
    high school; (4) the age of the winning candidate; (5) the log
    of the net assets of the winning candidate.
    \input{controls_etc_dfe}  }
  \setlength{\linewidth}{.1cm} \begin{center}
\newcommand{\contents}{\begin{tabular}{l*{5}{c}}
\hline\hline
 & BJP & INC & High School & Age & Log Net Assets \\ 
%PYTHON_HEADER
                    &         (1)   &         (2)   &         (3)   &         (4)   &         (5)   \\
\hline
Price shock$_{-6,-1}$&       0.009   &      -0.013   &       0.028   &      -2.443** &      -0.259   \\
                    &     (0.033)   &     (0.040)   &     (0.035)   &     (0.950)   &     (0.245)   \\
\hline State-Year F.E.      & Yes & Yes & Yes & Yes & Yes \\ 
 District F.E.   & Yes  & Yes  & Yes & Yes  &  Yes \\ 
\hline 
Mean Dep. Var. & 0.28 & 0.33 & 0.75 & 49.2 & 16.1 \\ 
%PYTHON_FOOTER
\hline
N                   &        2147   &        2147   &         915   &         946   &        1009   \\
r2                  &        0.40   &        0.31   &        0.31   &        0.33   &        0.48   \\
\hline
\multicolumn{6}{p{\linewidth}}{$^{*}p<0.10, ^{**}p<0.05, ^{***}p<0.01$} \\
\multicolumn{6}{p{\linewidth}}{\footnotesize \tablenote}
\end{tabular} }
\setbox0=\hbox{\contents}
\setlength{\linewidth}{\wd0-2\tabcolsep-.25em} \contents \end{center}

\end{table}

\newpage
\begin{table}[H]\caption{Effect of mineral price shocks on winning candidate
    criminality \cnewline by type of crime} \newcommand{\tablenote}{
    The table estimates the impact of a local mineral price shock on
    the criminality of the local elected leader, focusing on specific
    types of crime. The price shock is the change in global mineral prices, weighted by
constituency pre-sample production values of each mineral, calculated
over the five years preceding the given election.%
 In Column 1, the dependent
    variable is an indicator that takes the value one if the local
    election winner is facing charges for a violent crime, in which we
    include actual or attempted assault, armed robbery, homicide,
    kidnapping and sexual assault. In Column 2, we use an indicator
    that takes the value one if the election winner is charged with a
    non-violent crime, which is the set of crimes not used in
    Column 1. Column 3 estimates the impact on election winners being
    charged with corruption-related crimes (which include theft from
    government, manipulation of elections, and illegal influence over
    actions of public servants). Column 4 estimates the impact on
    election winners being charged with crimes other than those
    related to corruption. \input{controls_etc_dfe}} \small
  \setlength{\linewidth}{.1cm} \begin{center}
\newcommand{\contents}{\begin{tabular}{l*{4}{c}}
\hline\hline
 & \underline{Violent} & \underline{Non-violent} & \underline{Corruption} & \underline{Not Corruption} \\ 
%PYTHON_HEADER
                    &         (1)   &         (2)   &         (3)   &         (4)   \\
\hline
Price shock$_{-6,-1}$&       0.123***&      -0.042   &       0.037   &       0.043   \\
                    &     (0.042)   &     (0.044)   &     (0.040)   &     (0.050)   \\
\qquad
\textit{p-value
from
difference}
&
&
\textit{
0.02}
&
&
\textit{
0.94}
\\
\hline State-Year F.E. & Yes & Yes & Yes & Yes \\ 
District F.E. & Yes & Yes & Yes & Yes \\ 
\hline 
Mean Dep. Var. & 0.09 & 0.23 & 0.08 & 0.24 \\ 
%PYTHON_FOOTER
\hline
N                   &         935   &         935   &         935   &         935   \\
r2                  &        0.31   &        0.34   &        0.33   &        0.32   \\
\hline
\multicolumn{5}{p{\linewidth}}{$^{*}p<0.10, ^{**}p<0.05, ^{***}p<0.01$} \\
\multicolumn{5}{p{\linewidth}}{\footnotesize \tablenote}
\end{tabular} }
\setbox0=\hbox{\contents}
\setlength{\linewidth}{\wd0-2\tabcolsep-.25em} \contents \end{center}

  \label{tab:winner_violence}
\end{table}

\newpage
\begin{table}[H]\caption{Effect of mineral price shocks on election competitiveness}
  \newcommand{\tablenote}{The table estimates the impact of a local
    mineral price shock on several indicators of electoral
    competitiveness. All columns estimate Equation~\ref{eq:main} at the
    constituency-election year level. The price shock is the change in global mineral prices, weighted by
constituency pre-sample production values of each mineral, calculated
over the five years preceding the given election.%
 In Columns 1
    and 2, the dependent variable is an indicator that takes the value
    one if the local incumbent is re-elected. In Columns 3 and 4, the
    dependent variable is constituency level turnout. In Columns 5 and
    6, the dependent variable is the effective number of
    parties. Election data is available from 1990 to the
    present. Results are presented separately for elections for the full
    data from
    1990-2013 and from 2003-2013, a period comparable to other analyses
    in the paper. \input{controls_etc_dfe}} \footnotesize
  \setlength{\linewidth}{.1cm} \begin{center}
\newcommand{\contents}{\begin{tabular}{lcc|cc|cc}
\hline\hline
 & \multicolumn{2}{c}{\underline{Incumbent}} & \multicolumn{2}{|c}{\underline{Turnout}} & \multicolumn{2}{|c}{\underline{ENOP}} \\ 
%PYTHON_HEADER
                    &         (1)   &         (2)   &         (3)   &         (4)   &         (5)   &         (6)   \\
\hline
Price shock$_{-6,-1}$&      -0.036   &      -0.099   &       0.010   &       0.012   &       0.099*  &       0.092   \\
                    &     (0.047)   &     (0.074)   &     (0.007)   &     (0.017)   &     (0.055)   &     (0.143)   \\
\hline State-Year F.E. & Yes & Yes & Yes & Yes & Yes & Yes \\ 
District F.E. & Yes & Yes & Yes & Yes & Yes & Yes \\ 
Years & All & Post-2003 & All & Post-2003 & All & Post-2003 \\ 
\hline 
Mean Dep. Var. & 0.42 & 0.44 & 0.66 & 0.69 & 2.92 & 2.96\\ 
%PYTHON_FOOTER
\hline
N                   &        1617   &         625   &        1703   &         386   &        1695   &         473   \\
r2                  &        0.23   &        0.33   &        0.74   &        0.75   &        0.54   &        0.64   \\
\hline
\multicolumn{7}{p{\linewidth}}{$^{*}p<0.10, ^{**}p<0.05, ^{***}p<0.01$} \\
\multicolumn{7}{p{\linewidth}}{\footnotesize \tablenote}
\end{tabular} }
\setbox0=\hbox{\contents}
\setlength{\linewidth}{\wd0-2\tabcolsep-.25em} \contents \end{center}

  \label{tab:eci}
\end{table}

\newpage
\begin{table}[H]\caption{Effect of mineral price shocks on candidate asset
    growth and criminal activity} \newcommand{\tablenote}{The table
    shows estimates of the impact of mineral wealth shocks on asset
    growth of elected leaders, and on new criminal charges against
    them.  The dependent variable in columns 1-3 is the change in a
    candidate's log net assets over a single electoral term. The price
    shock is the unanticipated change in mineral wealth in that
    electoral term, defined as the change in the global prices of the
    basket of mineral in each constituency, measured from the first
    year after the politician is elected to the end of the electoral
    term. Column 1 estimates the regression on elected officials
    only. In Column 2, the sample includes winners and runners up from
    the first election, and the price shock is interacted with a
    dummy variable indicating the election winner.  Column 3 is analogous to column 1, but adds an
    interaction with politicians' criminal status in the baseline
    period, to test whether politicians already facing charges
    systematically gain more assets in response to a positive mineral
    wealth shock. Columns 4 and 5 run specifications comparable to
    Columns 1 and 2, where the dependent variable is an indicator for
    whether the politician is facing more criminal charges at the end
    of the electoral term than at the beginning. All regressions include state-year fixed effects and constituency
controls for the number of deposits within 10km of a constituency, a
constituency-level mineral dispersion index, and baseline (2001)
values of log constituency population, share of the population living
in rural areas, share of villages with electricity and the per
capita number of primary schools.  Standard errors are robust and
clustered at the district level.%
} \small
  \setlength{\linewidth}{.1cm} \begin{center}
\newcommand{\contents}{\begin{tabular}{lccc|cc}
\hline\hline
 & \multicolumn{3}{c}{\underline{Change in Assets}} & \multicolumn{2}{|c}{\underline{Change in Crime}} \\ 
%PYTHON_HEADER
                    &         (1)   &         (2)   &         (3)   &         (4)   &         (5)   \\
\hline
Price shock$_{+1,+5}$&       0.253** &      -0.072   &      -0.061   &       0.212***&      -0.045   \\
                    &     (0.104)   &     (0.172)   &     (0.183)   &     (0.061)   &     (0.085)   \\
Price shock$_{+1,+5}$ * Winner&               &       0.306*  &       0.214   &               &       0.244** \\
                    &               &     (0.169)   &     (0.191)   &               &     (0.100)   \\
Price shock$_{+1,+5}$ * Violent&               &               &       0.516   &               &               \\
                    &               &               &     (0.458)   &               &               \\
Price shock$_{+1,+5}$ * Winner * Violent&               &               &      -0.703   &               &               \\
                    &               &               &     (0.568)   &               &               \\
Violent Crime       &               &               &      -0.674   &               &               \\
                    &               &               &     (0.698)   &               &               \\
Winner              &               &      -0.256   &      -0.144   &               &      -0.389** \\
                    &               &     (0.254)   &     (0.283)   &               &     (0.158)   \\
Winner * Violent    &               &               &       1.018   &               &               \\
                    &               &               &     (0.866)   &               &               \\
\hline
State-Year F.E. & Yes & Yes & Yes & Yes & Yes \\ 
\hline 
Mean Dep. Var. & 1.02 & 0.98 & 0.99 & 0.18 & 0.20\\ 
%PYTHON_FOOTER
\hline
N                   &         448   &         696   &         583   &         364   &         629   \\
r2                  &        0.40   &        0.33   &        0.30   &        0.23   &        0.18   \\
\hline
\multicolumn{6}{p{\linewidth}}{$^{*}p<0.10, ^{**}p<0.05, ^{***}p<0.01$} \\
\multicolumn{6}{p{\linewidth}}{\footnotesize \tablenote}
\end{tabular} }
\setbox0=\hbox{\contents}
\setlength{\linewidth}{\wd0-2\tabcolsep-.25em} \contents \end{center}

  \label{tab:ts}
\end{table}
